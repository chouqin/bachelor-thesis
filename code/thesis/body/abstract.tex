\begin{abstract}

    随着互联网的不断发展,社交网络已经十分发达,逐渐成为了人们分享和交流信息的主要渠道。
    社交网络中的信息组成了一个个十分复杂的信息网络,
    在这些信息网络中,用户是一个个节点,而用户之间的联系则构成了节点之间的边。
    为了更好地分析这些信息网络的结构和功能,
    挖掘中隐藏在信息网络中的由关联密切的节点组成的社区是一个比较受关注的研究方向。

    尽管社区挖掘拥有如此高的重要性,面对现在膨胀式增长的信息网络,
    仍然面临着很多问题。其中最重要的问题是信息缺失,现在的信息网络无比庞大,
    要想收集整个信息网络的所有信息是不可能的。
    一般地,我们只能获取到少数的局部完整的信息网络,
    在这些局部完整的信息网络中,所有节点的信息都是已知且完整的。

    本文参考了目前最为热门的社区挖掘算法,结合不完全信息网络信息缺失的特点,
    提出一种信息缺失情况下的社区挖掘算法。
    首先,利用已知的局部完整信息网络学习到一个的距离度量,
    利用这个度量,可以求得所有节点之间的距离。
    然后,本文提出一种层次聚类方法,它能够基于节点之间的距离进行聚类,
    从而达到社区挖掘的目的。

    通过对比本文提出的算法和kmeans算法在基于Blogcatalog数据集下的进行社区挖掘的结果,
    发现本文所实现的算法相对于kmeans算法具有更高的精度,
    对于信息缺失网络下的社区挖掘具有更高的可靠性和实用性。

  \keywords{\large 社区挖掘 \quad 不完全信息网络 \quad 距离度量学习}
\end{abstract}

\begin{englishabstract}
    With the rapid development of the Internet, 
    social networks become more and more important in our daily lives.
    We have spent a lot of time sharing with our friends in social networks. 
    People in social networks, along with their relationships, 
    constitute an information network in which each person is a node 
    and their relationships are edges between nodes.
    Community detection is an important task
    for mining the structure and function of these information networks, and it has
    attracted much attention in the last decade.

    While community detection is very important, the task of collecting complete network
    data in social networks remains challenging because modern social networks are too large.
    Getting information for all the nodes in social networks is hard to achieve, 
    Usually we can get a small number local information regions, 
    Nodes in the local information region has complete information needed.

    In this paper, we proposed a community detection algorithm in incomplete social networks.
    We first learn a distance metric using the information in local information regions.
    We then use this distance metric to compute all the distance between any pair of nodes.
    A hierachical clustering approach is proposed to detect communities using node pairs'
    distance.

    The experimental results on real-world social networks show that, compared to kmeans algorithm, the algorithm 
    we proprosed has a much higher purity on community detection in community detection.

  \englishkeywords{\large Community Detection, incomplete information Networks, distance metric learning}
\end{englishabstract}
