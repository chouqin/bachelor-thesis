\chapter[英文大摘要]{Community Detection in Incomplete Social Networks}
\label{chap:longabstract}

Communities in social networks are groups of users which are densely
connected inside the group, while loosely connected with nodes outside the group.
Users belong to one community tends to have much more in common, 
they may share some common interests or have some common properties.
Detecting communities in social communities is very import to analyze
social networks.

Detecting communities in incomplete social networks has two meanings.
First, detecting communities in social networks can be viewed as an clustering problem,
of which there are many existing algorithms; 
Second, when the social network is incomplete, that is, there are some information missing
in the social networks, the problem gets much more tough.

A lot of works have been done to detecting communities in social networks,
Most of which focus on developing a clustering algorithm to do community detection.
Modularity-based clustering algorithm view the clustering process as an optimization problem,
which tries to find a clustering approach to maximize the modularity using some machine learning technicals. 
Density-based clustering algorithm find communities by taking region with high density as a cluster and 
taking region with low density as bound between clusters.
SHRINK combine these two method, using a hierarchical clustering algorithm to do clustering, meanwhile 
using modularity function to guide the whole process.
As for information networks with information missing, 
a distance metric can be learned by some distance metric learning algorithm,
then some distance based clustering algorithm can be applied to do community detection.

In this paper, we propose an algorithm that can detecting communities in incomplete social networks.
First, we identify and define the problem of incomplete networks.
Incomplete networks are information networks that have some information missing,
while there are many kinds of information missing regards to this problem,
in this paper, we deals with the incomplete information networks that still
has a few tiny local information regions. In these local regions, we can obtain
complete information of all nodes.

Second, we introduce a distance metric learning algorithm DCA to measure the 
distance between nodes in information networks. Using this distance metric,
we can compute the distance between any pair of nodes, which can reflect the
structural relation between nodes in incomplete networks. That is,
the distance between nodes that are similar to each other tends to be closer
than nodes that are dissimilar. To get such a distance metric, 
we turn the information in local information regions into 
contextual constraints which indicate the relevance relationship (positive or negative)
between nodes. The basic idea of DCA is to learn an
optimal data transformation that leads to the optimal distance 
metric by both maximizing the total variance between
the discriminative data chunklets and minimizing the total
variance of data instances in the same chunklets.

After such an distance metric has been learned, we can 
use some clustering algorithm to finish the community detecting process. 
In this paper, we propose a distance based clustering algorithm DSHRINK which can discover
the hierarchical communities. The DSHRINK clustering algorithm works as follows:
it iteratively combine nodes that are very closed to each other into one super nodes,
which can be combined in the next iteration. A distance based modularity function is devised
to evaluate the quality of the communities. Combination of low equality is stopped to 
get high purity of the clustering. We use this modularity to guide our clustering algorithm.
Moreover, so as to speedup the DSHRINK clustering process, we use an approximation strategy 
which leads to more nodes being combined during each iteration.

To evaluate the performance of our algorithm, we conduct a lot of experiments to compare
the purity of community detection using our methods and the kmeans clustering algorithm.
Empirical studies on real-world information networks show that our proposed
method can effectively detect community structures within incomplete social networks.

To conduct experiments, we first need to get proper datasets.
The datasets we choose must satisfy the following three conditions:
\begin{inparaenum}[i)]
    \item Each nodes must have proper attributes.
    \item There are some edges between nodes to generate local information regions.
    \item Each nodes must have class labels as ground truth to evaluate the result of community detection.
\end{inparaenum}
In order to get such datasets, we first implement an python crawler to 
crawl the data in the Weibo social networks. This crawler can get user's 
tweets, profiles, followers and fans. At the same time, we use The NLPIR to cut words
and extract keywords for each user, which can be used as attributes of an user.
But because the limitation of Weibo, we can't crawl enough user's information,
what's worse, most users in Weibo don't have class labels to use as ground truth.   
So we finally choose use data of Blogcatalog to do the experiments. 
PCA are used to reduce the dimension of attributes. Normalizing features
are required before doing PCA.
While the Weibo crawler we implemented can't be used to this project,
can be very helpful to get data for future work.

After proper datasets has been obtained, we need to generate contextual constraints 
to learn distance metric. To generate positive constraints, 
we use the structural similarity of nodes in the same local information regions.  
Local information regions are sampled using this procedure:
suppose we want to get regionNum local information regions, each local information has 
regionSize nodes, we first randomly pick one node as start node, then use BFS to include 
regionSize nodes from its neighbors, which is a local information region. The above procedure continues until we have sampled 
regionNum local information regions. Negative constraints are generated using 
randomly sampled regionNum times regionSize nodes.

After the above process, using DCA to learn a distance metric, followed by DSHRINK to clustering nodes 
based on the distance learned, we finish the whole procedure of our community detection method.
In order to measure the effectiveness of our method, we adopt purity to evaluate the quality of the 
communities generate by different approaches. The definition of purity is as follows:
each cluster is first assigned with the class that has the most nodes in the cluster as it's label,
number of nodes this label contains is the cluster's label count,
and purity is defined by the sum of all clusters' label count divided by total number of nodes, 
or more formally, in equation \ref{equa:purity}. Using this purity, 
we compare our methods with kmeans clustering algorithm using different datasets 
and under different conditions. Because of randomness of our method, 
the result of each run may be significantly different even under the same conditions.  
To overcome this randomness, each experiment is repeated 5 times and take the 
average result as the final result.
An automation script is implemented to execute this procedure automatically.  

From result section \ref{sec:results} we can see that the method we proposed in this paper
both have effectiveness and efficiency over kmeans.
This is because our method not only use nodes' attributes
to compute distance, but also use 
the information in local information region as side knowledge.
In the result we can see, 
the larger the local region information's size,
or the more the local region information's number,
the more side knowledge we can obtain,
which results in the higher purity.
What's more, 
our approach can automatically detect the appropriate number of cmmunities by minimizing the 
distance based modularity. Since the number of communities is unknown, 
kmeans method has great limitation because it must number of communities k must be known to
use kmeans clustering. Choose the appropriate k is an hard problem which needs a lot of more 
efforts. our method has great advantage for detecting communities than kmeans.

The method we propose to detecting communities in incomplete networks 
can not only be used in social networks, but also a wide range of domains.
For example, we can use our method to biological networks, publication networks,
terrorist-attack networks, food web, and so on. 
It is very interesting to analyze these networks using our method. 
