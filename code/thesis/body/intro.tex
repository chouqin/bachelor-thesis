\chapter{绪论}
\label{chap:intro}

\section{课题背景}

社交网络是互联网高速发展的产物,作为Web2.0时代的标志性产品,
社交网络推动了计算机产业的又一波浪潮。国外的Facebook, Twitter等社交网络的崛起,
造就了互联网的一个个神话。而在国内,微博和人人是比较典型的网络,
这些社交网络逐渐地拉近了人们之间的联系,成为网民交流信息的主要渠道。
目前,约有一半以上的中国网民通过社交网络沟通交流、分享信息,
社交网络已成为覆盖用户最广、传播影响最大、商业价值最高的Web2.0业务。

通过观察社交网络,我们发现,在用户和用户之间,有的联系比较紧密,
他们或者关注同样的话题,有共同的兴趣爱好,或者他们有共同的好友,
经常在社交网络上进行交流,我们把关系比较密切的一组用户称为一个社区。
社交网络可以看成由一个个的社区组成,同一个社区里的用户关系紧密,而不同社区的用户则彼此之间关系不大。
社区这个概念在现在的社会社交网络中已经越来越明显,比如国外的Google+,用户可以把自己的好友划分为一个个的“圈子”,
这样的“圈子”就可以看成是一个社区,每一个圈子里的人具有较多的共同点。又比如在新浪微博中,
用户可以把自己的关注对象分到一个个自己设定好的“分组”里面,这样的一个“分组”也可以看成是一个社区。

在社交网络中,挖掘出用户组成的社区对于分析用户行为,研究用户关系显得十分重要。
如果我们能够把两个用户划分到同一个社区,那么我们就知道这两个用户肯定有很多相似的地方或者属于同一个领域。
我们可以认为他们之间的关系比较紧密,甚至可以根据其中一个用户的行为来预测另外一个用户的行为。
对于想利用社交网络进行精准营销的广告主来说,可以根据一个用户的行为来对另外一个用户进行推荐。
同时,如果我们想找到某一个领域的一群人,我们只需要找到由这样的用户组成的社区即可。

虽然在社交网络中进行社区挖掘拥有如此高的重要性,可它一直都是一个很复杂的问题。
它的复杂性一方面来自于社交网络的高速增长,每天有几十亿的人活跃于各个社交网络,
有大量的信息被产生和分享,社交网络也因此变得十分庞大,面对一个如此庞大的信息网络,
必须设计出一种和普通算法不一样的算法。同时,在对社交网络进行分析时,信息缺失也是一个值得考虑的问题,
我们无法保证我们能够获得整个社交网络的全部信息,只能根据收集到的部分信息对社交网络进行研究。
在这样信息缺失的情况下对社交网络进行社区挖掘,显然是更有挑战性和实际意义的。

在本文中,我们提出一种在信息缺失的信息网络中进行社区挖掘的算法。
我们的算法对应的是这样一种信息缺失的信息网络:在这种信息网络中,
虽然我们不能所有用户的完整信息,但是我们能够得到少数几个局部信息网络,
在每一个局部信息网络中的任何用户的信息是完整的。这样的假设显然是与实际情况十分吻合的,
在现在的社交网络中,得到所有用户的信息是不现实的,可得到少数几个由用户组成的局部区域显然是可行的。
基于这样的假设,本文提出了一种社区挖掘的算法。

首先,我们提出一种距离度量学习(Distance metric learning)算法,
该算法能够利用已知的局部网络的信息学习一个距离度量(Distance metric),
根据这个距离度量我们可以计算网络中任意两个节点之间的距离。
然后,我们提出一种聚类算法,该算法能够利用节点之间的距离对其进行聚类,从而达到社区挖掘的目的。

\section{文献综述}

社区挖掘这个研究方向在各个领域有着广泛的研究,有很多聚类算法都可以用来解决特定情况下的社区挖掘问题。
基于模块性准则的聚类算法(Modularity-based methods)是一种机器学习算法,它以模块性作为评价聚类好坏的标准,
以它作为目标函数,利用机器学习得到这个目标函数的最优解,此时的聚类具有最好的模块性,也就是具有最好的质量\upcite{newman2004finding,aggarwal2011towards,guimera2005functional,newman2004fast}。给定一个信息网络,它被划分为$k$个社区,模块性函数$Q$定义为:

\begin{equation}
Q = \sum_{i=1}^{k}[ \frac{l_i}{L} - (\frac{d_i}{2L})^2]
\end{equation}

其中$L$是整个网络中边的条数,$l_i$是社区$i$中边的条数,$d_i$是社区$i$中所有节点度的总和。
通过求解$Q$的最优值,能够得到最优的聚类的结果。通常,求解$Q$的最优化问题是一个NP困难问题,
因此,有许多贪婪的算法用于得到近似的最优解。

基于密度的聚类方法(Density-based methods)基于节点的疏密程度对节点进行聚类\upcite{bortner2010progressive,ester1996density,xu2007scan},
它把节点密度较大的区域作为一个聚类,而把节点密度较小的区域作为聚类之间的分界区。
SHRINK结合了上述两种方法的优点,
利用一种层次化的聚类方法不断地把结构相似的节点聚合到一起形成一个超级节点,
同时使用模块性准则作为迭代终止的判别标准\upcite{Huang:2010:SSC:1871437.1871469}。
而且SHRINK算法能够检测出信息网络中的孤立节点。

随着不完整信息网络的出现,解决社区挖掘问题变得更加复杂。利用已知的部分网络来对整个网络进行推测一定程度上解决了这个问题\upcite{kim2011network}。
而更加一般的办法是利用已知的信息学习一个距离度量,
使用这个度量可以求得所有节点对之间的距离\upcite{xing2002distance,hoi2006learning,bar2006learning}。
学习距离度量的目的是找到一个转换矩阵$M$,使得转换之后的距离能够保证相似的节点之间的距离更近。
这就可以把距离度量的学习问题转换为一个最优化问题,比如,如果已知$S$是相似的节点列表(任何$(u, v) \in S$则$u, v$是相似的),
$D$是不相似节点列表(任何$(u, v) \in D$则$u, v$是不相似的),则求解$M$可以转换为以下最优化问题:

\begin{equation}
\begin{aligned}
& \underset{M}{\min}
& & \sum_{(u,v) \in S} (\sqrt{(u-v)^\top M (u-v)})^2 \\
& \text{s.t.}
& & \sum_{(u,v) \in D} \sqrt{(u-v)^\top M (u-v)} \geq 1 \\
& & & M \geq 0
\end{aligned}
\end {equation}

这个最优化问题是一个凸型最优化问题,可以用最优化方法解决,比如说二次规划法,半正定规划法。
通过求解这个最优化问题,我们可以学习到一个距离距离度量,
利用距离度量可以计算节点之间的距离,就可以基于距离进行聚类,
从而达到社区挖掘的目的\upcite{Lin:2012:CDI:2187836.2187883}。

\section{社区挖掘在其他领域的应用}

尽管在本文中,我们主要关注在信息缺失的社交网络中进行社区挖掘,社区挖掘在其他的领域也得到广泛应用。
因为社区挖掘是针对信息缺失的信息网络的,对于各种各样的信息网络都能使用社区挖掘进行分析。
一些比较典型的信息缺失网络有:

\begin{itemize}
    \item 恐怖主义袭击网络。恐怖主义袭击网络由一系列的恐怖主义所引发的事件组成的,
    每一个事件可以看成信息网络的一个节点,如果两个事件是由同一个恐怖主义者执行的,
    那么这两个节点之间有一条边。这样的网络是一种信息缺失网络,
    因为我们无法得知所有恐怖主义事件的触发者,但是我们能够得到少数几个区域的完整信息。
    如果能够将社区挖掘应用到这样的网络中,对于分析和预测恐怖主义的行为将显得非常重要。
    \item 食物网络。大型生态系统的是一个非常复杂的网络,在这个网络中每一个节点代表一个物种,
    网络中的边代表物种之间的捕食关系。很显然,要获取整个食物网的全部信息是不可能的,但是,
    我们能够比较容易地得到少数几个局部的食物网络的信息。
    在这样的食物网络中进行社区挖掘能够帮助我们找到小型的生态系统,以及在这些生态系统中生活的物种。
\end{itemize}

\section{本文创新点}

总的来说,本文有如下创新点:

\begin{enumerate}
\item 提取出信息缺失网络的一种特定场景,在本文中的信息缺失网络拥有一些局部完整网络,局部完整网络中的节点的信息是完整的。
\item 实现在上述信息缺失网络下的距离度量学习算法,通过这个距离度量得到的距离能够保证已知的比较相似的节点距离较近。
\item 实现一种基于距离的社区挖掘算法,该算法具有较好的效果。
\end{enumerate}

\section{本文研究目标和内容}

本文的研究目标是设计出一种在信息缺失情况下进行社区挖掘的算法。
主要工作集中在以下几个方面:

\begin{enumerate}
\item 选取合适的数据集,保证数据集中的节点有合适的属性,节点之间有边,同时每个节点有标签信息用于评价最后社区挖掘的结果。
\item 构造信息缺失网络,使信息网络中存在一些局部信息网络,局部信息网络中的所有节点的信息是完整的。
\item 实现一个距离度量学习算法,能够利用局部信息网络的信息学习一个距离度量,保证相似的节点之间的距离更近。
\item 实现一个基于距离的聚类算法,基于节点之间的距离进行聚类,最终达到社区挖掘的目的。
\item 设计并完成实验,验证本文提出算法的有效性。
\end{enumerate}

\section{本文结构安排}

对于本文的结构做如下安排:

\begin{itemize}
    \item 在第\ref{chap:back}章中,介绍本文提出的算法所涉及的背景知识,主要包含距离度量和聚类算法两部分。
        其中距离度量是本文提出的距离度量学习算法的理论基础,而对于各种聚类算法的分析和比较是提出本文提出的基于距离的聚类算法的来源。
    \item 在第\ref{chap:dca}章中,我们详细介绍本文提出的距离度量学习算法的机理和实现细节,利用学习到的距离度量,
        我们可以计算所有节点之间的距离,这个距离能够保证相似的节点之间距离更近。
    \item 在第\ref{chap:dshrink}章中,我们将阐述本文提出的基于距离的聚类算法——DSHRINK,并提出一种近似的手段能够加快算法的运行。
    \item 在第\ref{chap:implementation}章中,我们使用本文提出的算法与kmeans算法进行比较,通过大量的实验验证本文算法的有效性。
    在实验的过程中,我们解决了如何选取合适的数据集,如何生成局部信息网络,如何自动化实验和如何评估算法的有效性等一系列的问题。
    \item 第\ref{chap:summary}章对本文进行总结与展望。
\end{itemize}
