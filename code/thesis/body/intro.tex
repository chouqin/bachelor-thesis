\chapter{绪论}
\label{chap:intro}

\section{课题背景}

在社交网络中,挖掘出用户组成的社区对于分析用户行为,研究用户关系显得十分重要。
对于社交网络来说,处在同一个社区里的用户之间联系紧密,他们拥有更多的共同爱好,而处在不同社区的用户则彼此之间关系不大。
社交网络可以看成一个大型的信息网络,其中用户是信息网络的节点,而用户之间的关系可以看成信息网络的边。
由于现在的社交网络十分庞大,要获取整个社交网络的信息是不可能的,通常情况下我们只能获取少数几个局部区域,在每一个局部区域里用户的信息是完整的。

在本文中,我们提出一种在这样的信息缺失网络中挖掘社区的算法。
首先,我们利用已知的局部网络的信息学习一个距离度量(Distance metric),根据这个距离度量我们可以计算网络中任意两个节点之间的距离。然后,我们使用一种基于距离的层次化聚类方法DSHRINK,从而达到社区挖掘的目的。

社区发现这个研究方向在各个领域有着广泛的研究,有很多聚类算法都可以用来解决特定情况下的社区发现问题。
基于模块性准则的聚类算法(Modularity-based methods)是一种机器学习算法,它以模块性作为评价聚类好坏的标准,通过学习得到一个最优的模块性的聚类\upcite{newman2004finding,aggarwal2011towards}。
基于密度的聚类方法(Density-based methods)基于节点的疏密程度对节点进行聚类\upcite{bortner2010progressive,ester1996density,xu2007scan}。
SHRINK结合了上述两种方法,通过迭代式地基于结构相似度将节点聚合成一个超级节点,同时用模块性准则作为迭代终止的判别方法\upcite{Huang:2010:SSC:1871437.1871469}。
随着不完整信息网络的出现,解决聚类问题变得更加复杂。利用已知的部分网络来对整个网络进行推测一定程度上解决了这个问题\upcite{kim2011network}。
而更加一般的办法是利用已知的信息学习一个距离度量(Distance metric learning),
使用这个度量可以求得所有节点对之间的距离\upcite{xing2002distance,hoi2006learning,bar2006learning}。
求出这个距离之后,就可以基于距离进行社区挖掘\upcite{Lin:2012:CDI:2187836.2187883}。

总的来说,本文有如下的贡献:

\begin{enumerate}
\item 提取出信息缺失网络的一种特定场景,在本文中的信息缺失网络拥有以下局部完整网络,局部完整网络中的节点的信息是完整的。
\item 实现在上述信息缺失网络下的距离度量学习算法,通过这个距离度量得到的距离能够保证已知的比较相似的节点距离较近。
\item 实现一种基于距离的社区挖掘算法,该算法具有较好的效果。
\end{enumerate}

\section{本文研究目标和内容}

本文的研究目标是设计出一种在信息缺失情况下进行社区挖掘的算法。
主要工作集中在以下几个方面:

\begin{enumerate}
\item 选取合适的数据集
\item 构造信息缺失网络
\item 实现一个距离度量学习算法
\item 实现一个基于距离的聚类算法
\item 完成实验
\end{enumerate}

\section{本文结构安排}

本文接下来的部分将做如下安排:第\ref{chap:dca}章介绍本文实现的距离学习算法,第\ref{chap:dshrink}章介绍DSHRINK算法,第\ref{chap:implementation}章介绍整个算法的实现与实验的步骤,第\ref{chap:result}章介绍实验结果并分析讨论,第\ref{chap:summary}章对本文工作进行总结与展望。
